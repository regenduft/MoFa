\subsection*{Mofa::Model::Point\label{Mofa::Model::Point}\index{Mofa::Model::Point}}


Objekte dieser Klasse repr�sentieren einen Punkt auf der Oberfl�che der Erde.
Erbt von: \texttt{Mofa::Model::Object}

\minisec{Beschreibung\label{Beschreibung}\index{Beschreibung}}


This class represents a point on the earth. The coordinates can be referred
as latitude-longitude dezimal coordinates with \texttt{Mofa::Model::Point::ll()} or as utm-coordinates
in meters with \texttt{Mofa::Model::Point::utm()}. These methods read without parameter or set with
2 (or 4 in case of utm) parameters. You should only use WGS84-coordinates.

\minisec{Klassenmethoden\label{Klassenmethoden}\index{Klassenmethoden}}
\minisec{\texttt{\$ new(\$)}\label{_new_}\index{\$ new(\$)}}


\textit{Parameter:} Hashreferenz



\textit{R�ckgabewert:} \texttt{Point}-Objekt



\textit{Beispiele:}

\begin{verbatim}
   $pt = 
       Mofa::Model::Point->new({
           utm_x=>413794, utm_y=>5470842, 
           utm_e=>32,     utm_n=>"U"
       });
   $pt = Mofa::Model::Point->new({
           utm_x=>413794, utm_y=>5470842
       });
   # 32, "U" will be used as values for utm_e and utm_n.
\end{verbatim}
\begin{verbatim}
   $pt = Mofa::Model::Point->new({x=>49.39, y=>8.81});
   $pt = Mofa::Model::Point->new({
             coord=>{X=>"49 25 37N", Y=>"7 45 02E"}
         });
\end{verbatim}
\minisec{\texttt{(\$\$) dms2dez(\$\$)}\label{_dms2dez_}\index{(\$\$) dms2dez(\$\$)}}


\textit{Parameter:} Koordinaten in Grad Minute Sekunde (\texttt{string}, \texttt{string})



\textit{R�ckgabewert:} Koordinaten als Dezimalzahl (\texttt{float}, \texttt{float})



\textit{Beispiel:}

\begin{verbatim}
   ($x, $y) = 
     Mofa::Model::Point::dms2dez("49 23 03N", "7 48 44E");
\end{verbatim}
\minisec{\texttt{(\$\$) utm2ll(\%)}\label{_utm2ll_}\index{(\$\$) utm2ll(\%)}}


\textit{Parameter:} Hash mit Attributen utm\_x, utm\_y, utm\_e, utm\_n



\textit{R�ckgabewert:} Koordinaten als Dezimalzahl (\texttt{float}, \texttt{float})



\textit{Beispiel:}

\begin{verbatim}
   ($x, $y) = 
       utm2ll({
           utm_x=>413794, utm_y=>5470842, 
           utm_e=>32,     utm_n=>"U"
       });
\end{verbatim}
\minisec{\texttt{(\$\$\$\$) ll2utm(\%)}\label{_ll2utm_}\index{(\$\$\$\$) ll2utm(\%)}}


\textit{Parameter:} Hash mit Attributen x, y



\textit{R�ckgabewert:} Utm-Koordinaten



\textit{Beispiel:}

\begin{verbatim}
   ($meters_x, $meters_y, $square_e, $square_n) 
       = ll2utm({x=>49.39, y=>8.81});
\end{verbatim}
\minisec{Objektmethoden\label{Objektmethoden}\index{Objektmethoden}}
\minisec{\texttt{(\$\$\$\$) utm(@)}\label{_utm_}\index{(\$\$\$\$) utm(@)}}


\textit{Parameter:} [utm\_x (\texttt{int}), utm\_y (\texttt{int}), [utm\_e (\texttt{int}), utm\_n (\texttt{string})]]



\textit{R�ckgabewert:} (utm\_x, utm\_y, utm\_n, utm\_e)



\textit{Beispiele:}

\begin{verbatim}
   Lesen:  ($utm_x, $utm_y)   = $pt->utm;
   Setzen: $pt->utm(413794, 5470842, 32, 'U');
           $pt->utm(413794, 5470842);
\end{verbatim}
\minisec{\texttt{(\$\$) ll(@)}\label{_ll_}\index{(\$\$) ll(@)}}


\textit{Parameter:} [x (\texttt{int}), y (\texttt{int})]



\textit{R�ckgabewert:} (x, y)



\textit{Beispiele:}

\begin{verbatim}
   Lesen:  ($x, $y) = $pt->ll;
   Setzen: $pt->ll(49.32, 7.32);
\end{verbatim}
